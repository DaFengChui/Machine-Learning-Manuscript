

\documentclass[11pt]{book}

% if chinese
% -----------------------------------------------------------------
 % \usepackage[UTF8]{ctex}


% page setup
% -----------------------------------------------------------------
\usepackage[b5paper,top=1.3in,bottom=1.3in]{geometry}  % Document margins


% \usepackage{booktabs}             % Horizontal rules in tables


% header footer
% -----------------------------------------------------------------
\usepackage{fancyhdr}          % Headers and footers
  \pagestyle{fancy}            % All pages have headers and footers
  \fancyhead{}                 % Blank out the default header
  \fancyfoot{}                 % Blank out the default footer
  \fancyhead[C]{\small Manuscripts $\bullet$ \today $\bullet$} % Custom header text
  \fancyfoot[RO]{\thepage}     % Custom footer text, control pagenumber location
  %\fancyfoot[RO,LE]{\thepage} % Custom footer text

% graphicx 
% -----------------------------------------------------------------
\usepackage{graphicx}
\usepackage{subfigure}
\usepackage{float}                        % [H] not float
\usepackage{tikz}
\usepackage[hang,small,up,up,labelfont=bf,textfont=it]{caption} % Custom captions under/above floats in tables or figures


% color
% -----------------------------------------------------------------
\usepackage{xcolor}


% Algorithm
% -----------------------------------------------------------------
\usepackage{listings}
\usepackage{algpseudocode}
\usepackage{fancyvrb}                     % verbatim env  抄录


% customized lists
% -----------------------------------------------------------------
\usepackage{enumitem}                     % Customized lists
  \setlist[itemize]{noitemsep}            % Make itemize lists more compact


% mathematics
% -----------------------------------------------------------------
\usepackage{amsmath,amssymb,amsfonts,amsthm}
\allowdisplaybreaks
\newtheorem{theorem}{Theorem}[section]
\newtheorem{lemma}{Lemma}[section]
\newtheorem{corollary}[theorem]{Corollary}
\newtheorem{proposition}[theorem]{Proposition}
\newtheorem{definition}{Definition}[section]
\newtheorem{example}{Example}
\newtheorem{remark}{Remark}

% label equation with section 1,2,3,...
\makeatletter                             % `@' now normal "letter"
\@addtoreset{equation}{section}
\makeatother                              % `@' is restored as "non-letter"
\renewcommand\theequation{\oldstylenums{\thesection}.\oldstylenums{\arabic{equation}}}


% hyperlinks
% -----------------------------------------------------------------
\usepackage{bookmark}
\usepackage{hyperref}
  \hypersetup{colorlinks,
              citecolor=black,
              filecolor=black,
              linkcolor=black,
              urlcolor=black
              }


% user-defined
% -----------------------------------------------------------------
% \newcommand{\R}{\mathbb{R}}




% -----------------------------------------------------------------
% TITLE SECTION
% -----------------------------------------------------------------

\title{Learning Manuscripts}                                                     % Article title
\author{                                                                         % authors 1
  \textsc{JY. J}
  \normalsize ECNU \\                                                            % institution
  \normalsize \href{jy.jiang@outlook.com}{jy.jiang@outlook.com}                  % email address
}
\date{\today}





\begin{document}
\maketitle
% \thispagestyle{empty}  % no page number
\tableofcontents

% \setcounter{page}{0}
% \thispagestyle{empty}



% \chapter{Introduction}


\section{What is ML?}


\section{Classification of ML}






% 
\chapter{Machine Learning Foundation}

\section{Learning Algorithm}

\section{Capacity and Fitting}

\section{Hyper Parameter}


\section{Maximum Likelihood Estimation}


\section{Bayesian Statistics}


\section{Supervised Learning}


\section{Unsupervised Learning}

\subsection{The Challenge of Unsupervised Learning}
Unsupervised learning is often much more challenging than supervised learning. Because its exercise tends to be more subjective, and there is no simple goal for the analysis, such as prediction of a response. 

Unsupervised learning is often performed as part of an exploratory data analysis. It can be exploratory hard to assess the results of unsupervised learning methods, since there is no universally accepted mechanism for performing crossvalidation or validating results on an independent data set. 


\section{Basic Descent Methods}
\subsection{Fibonacci and Golden section Search}
\subsection{Line search by Curve Fitting}
\subsection{The Method of Line Search Algorithms}
\subsection{Newton's Method}
\subsection{Coordinate Descent Method}


\section{Construction of Machine Learning Algorithm}






















% \chapter{Data preprocessing}

this chapter we are going to 

\section{Overview of Data Preprocessing}

\section{Data Cleaning}

\section{Dimensionality Reduction}

\section{Data Transformation}

Standardization
Non-linear Transformation
Normalization
Binarization
Encoding Categorical Features
Imputation of Missing Values
Generating Polynomial Features
Custom Transformers

 














% \chapter{Dimensionality Reduction}




\newpage
\section{Principal Components Analysis}
PCA is an unsupervised approach. When faced with a large set of correlated variables, principal components allow us to summarize this set with a smaller number of representative variables that collectively explain most of the variability in the original set.

\subsection{Definition of PCA}
Assume that $\mathbf{x} = (x_1,x_2,\cdots,x_m)$ is the $m$-dim random variables, with $\mathbf{E}(\mathbf{x}) = \mathbf{\mu},~\mathbf{Cov}(\mathbf{x}) = \mathbf{\Sigma}$. 

Consider that the projection of $\mathbf{x}$ on the direction $\mathbf{w}$ is
\[
	z = \mathbf{w}^T \mathbf{x}
\]

The $k-th$ principal component is the $\mathbf{w_k}$, which satisfies the projection of the samples $\mathbf{x}$ on $\mathbf{w}_k$, and its variance in order $k$. In order to obtain unique solution, we require $\|\mathbf{w}_k\|_2^2 = 1$. If we denote $z_k = \mathbf{w}_k^T\mathbf{x}$, then
\[
	Var(z_k) = \mathbf{w}_k^T \mathbf{\Sigma} \mathbf{w}_k
\]

Let us look for the first principal component
\[
	\underset{\mathbf{w}_1}{\min\,} \mathbf{w}_1^T \mathbf{\Sigma} \mathbf{w}_1 + \alpha_1(1-\mathbf{w}_1^T\mathbf{w}_1 )
\]
and the cost function is 
\[
	J(\mathbf{w}_1) = \mathbf{w}_1^T \mathbf{\Sigma} \mathbf{w}_1 + \alpha_1(1-\mathbf{w}^T\mathbf{w} )
\]
we set its derivatives to zero, and obtain 
\[
	2\mathbf{\Sigma} \mathbf{w}_1 - 2\alpha_1\mathbf{w} = 0,\quad \text{thus } \mathbf{\Sigma} \mathbf{w}_1 = \alpha_1\mathbf{w}.
\]

\noindent If $\mathbf{w}_1$ is the eigenvector of $\mathbf{\Sigma}$, and $\alpha_1$ is the corresponding eigenvalue, then the upper form is established. In order to make the variance reach the max, $\mathbf{w}_1$ should be the eigenvector corresponding to the largest eigenvalue $\alpha_1 = \lambda_1$.

Now, let us look for the first principal component

\noindent Similarly, the second principal component $\mathbf{w}_2$  should also maximize variance, and satisfies $\|\mathbf{w}_2\|_2^2 = 1$, $\mathbf{w}_2^T\mathbf{w}_1 = 0$. Consider
\[
	\underset{\mathbf{w}_2}{\min\,} \mathbf{w}_2^T \mathbf{\Sigma} \mathbf{w}_2 + \alpha_2(1-\mathbf{w}_2^T\mathbf{w}_2) + \beta(0-\mathbf{w}_2^T\mathbf{w}_1)
\]
the cost function is 
\[
	J(\mathbf{w}_1) = \mathbf{w}_2^T \mathbf{\Sigma} \mathbf{w}_2 + \alpha_2(1-\mathbf{w}_2^T\mathbf{w}_2) + \beta(0-\mathbf{w}_2^T\mathbf{w}_1)
\]
we set its derivatives to zero, then obtain 
\begin{equation}
	\label{eq:pca_2nd}
	2\mathbf{\Sigma} \mathbf{w}_2 - 2\alpha_2\mathbf{w}_2 - \beta\mathbf{w}_1 = 0.
\end{equation}
left multiply with $\mathbf{w}_1^T$, get
\[
	2\mathbf{w}_1^T\mathbf{\Sigma} \mathbf{w}_2 - 2\alpha_2\mathbf{w}_1^T\mathbf{w}_2 - \beta\mathbf{w}_1^T\mathbf{w}_1 = 0.
\]
Note that $\mathbf{w}_1^T\mathbf{w}_2=0$, $\mathbf{w}_1^T\mathbf{\Sigma} \mathbf{w}_2 = \mathbf{w}_2^T\mathbf{\Sigma} \mathbf{w}_1 = \lambda_1\mathbf{w}_2^T\mathbf{w}_1 = 0$, thus $\beta = 0$

together with \ref{eq:pca_2nd}, we can get
\[
	\mathbf{\Sigma} \mathbf{w}_2 = \alpha_2\mathbf{w}_2.
\]
thus $\mathbf{w}_2$ also should be the eigenvector of $\mathbf{\Sigma}$, and $\alpha_1$ is the corresponding second largest eigenvalue, i.e. $\alpha_2 = \lambda_2$.

Similarly, we can get all the $k-th$ principal components in turn.

Actually, the decomposition process of PCA is the spectral decomposition process of $\Sigma$, with eigenvalue in order $\lambda_1 \geq \cdots \geq \lambda_m \geq 0$(because $\Sigma$ is nonnegative definite), and the corresponding eigenvector is $\mathbf{w}_1,\cdots,\mathbf{w}_m$

if we denote 
\begin{align*}
	& \mathbf{W} = (\mathbf{w}_1,\cdots,\mathbf{w}_m)\\
	& \Lambda = diag(\lambda_1,\cdots,\lambda_m)
\end{align*}
then
\[
	\mathbf{\Sigma} = \mathbf{W}\Lambda\mathbf{W}^T = \sum_{i=1}^{m}\lambda_i \mathbf{w}_i\mathbf{w}_i^T
\]
Thus
\[
	\mathbf{w}_j^T\mathbf{\Sigma}\mathbf{w}_j =
	 (\mathbf{W}^T\mathbf{w}_j)^T\Lambda \mathbf{W}^T\mathbf{w}_j =
	 \sum_{i=1}^{m}\lambda_i \mathbf{w}_j^T\mathbf{w}_i(\mathbf{w}_j^T\mathbf{w}_i)^T= \sum_{i=1}^{m}\lambda_i (\mathbf{w}_j^T\mathbf{w}_i)^2=\lambda_j
\]
if we define 
\begin{equation}
	\label{eq:pca_relation}
	\mathbf{y} = \mathbf{W}^T \mathbf{x}
\end{equation}
then \ref{eq:pca_relation} is the relation between $\mathbf{x}$ and principal component vector $\mathbf{y}$.

For understanding PCA better, let us consider $\mathbf{x} \backsim \mathcal{N}_m(\mathbf{\mu,\Sigma})$, then consider the density equal surface of $\mathbf{x}$
\begin{equation}
	\label{eq:pca_sphere}
	(\mathbf{x}-\mathbf{\mu})^T\mathbf{\Sigma}^{-1}(\mathbf{x}-\mathbf{\mu}) = c^2, \quad c>0.
\end{equation}
thus
\begin{align*}
	c^2 &= (\mathbf{x}-\mathbf{\mu})^T\mathbf{\Sigma}^{-1}(\mathbf{x}-\mathbf{\mu})\\
	&= [\mathbf{W}^T(\mathbf{x}-\mathbf{\mu})]^T\mathbf{\Lambda}^{-1}[\mathbf{W}^T(\mathbf{x}-\mathbf{\mu})]\\
	&= (\mathbf{y}-\mathbf{v})^T\mathbf{\Lambda}^{-1}(\mathbf{y}-\mathbf{v})\\
	&= \frac{(y_1-v_1)^2}{\lambda_1}+\frac{(y_2-v_2)^2}{\lambda_2}+\cdots+\frac{(y_m-v_m)^2}{\lambda_m}
\end{align*}
where $\mathbf{v} = (v_1,\cdots,v_m)^T = \mathbf{E}(\mathbf{y}) = \mathbf{W}^T\mathbf{\mu}$. \ref{eq:pca_sphere} is a ellipsoid centered on $(v_1,\cdots,v_m)$, the axis direction is $(\mathbf{w}_1,\cdots,\mathbf{w}_m)$, and the radius is $(c\sqrt{\lambda_1},\cdots,c\sqrt{\lambda_m})$.


\subsection{Properties}
\begin{enumerate}
\item Covariance matrix of principal component vector
	\[\mathbf{Cov}(\mathbf{y}) = \mathbf{\Lambda}\]
	where $\mathbf{\Lambda} = diag(\lambda_1,\cdots,\lambda_m)$, i.e. $\mathbf{Cov}(y_i) = \lambda_i$, and $y_1,\cdots,y_n$ irrelevant to each other.

\item Total variance of principal component
	\[	
		trace(\mathbf{Cov}(\mathbf{x})) = trace(\mathbf{Cov}(\mathbf{y})),~i.e.~ \sum_i^m \sigma_{ii} = \sum_i^m \lambda_i 
	\]
	The ratio of variance of j-th principal component to total variance
	\[	
		\frac{\lambda_j}{\sum_{i=1}^{m}\lambda_i}
	\]
	we call it \textbf{contribution rate} of the principal component.

	Thus, the \textbf{cumulative contribution rate} of top $k$ principal component was defined as 
	\[	
		\frac{\sum_{j=1}^{k}\lambda_j}{\sum_{i=1}^{m}\lambda_i}
	\]
	It shows the ability of $y_1,\cdots,y_k$ to explain $x_1,\cdots,x_m$

\item the relationship between orignal variable $\mathbf{x}$ and principal component variable $\mathbf{y}$
	\[	
		\mathbf{x} = \mathbf{Wy}
	\]
	i.e.
	\[	
		x_i = w_{i1}y_1 + \cdots + w_{im}y_m,~i = 1,\cdots,m
	\]
	thus
	\begin{align*}
		& \mathbf{Cov}(x_i,y_j) = \mathbf{Cov}(w_{ij}y_j,y_j) = w_{ij}\lambda_j\\
		& \rho(x_i,y_j) = \frac{\mathbf{Cov}(x_i,y_j)}{\sqrt{\mathbf{Var}(x_i)\mathbf{Var}(y_i)}} = \sqrt{\frac{\lambda_j}{\sigma_{ii}}}w_{ij},~i,j = 1,\cdots,m
	\end{align*}

\item the contribution rate of top $k$ principal component $y_1,\cdots,y_k$ to the original variable $x_i$

	We use the \textbf{complex correlation coefficient} $\rho_{i\dot,1,\cdots,k}^2$ to measure the contribution of the top $k$ principal component to the original variable $x_i$.
	\[
		\rho_{i\cdot 1,\cdots,k}^2 = \sum_{j=1}^{k}\rho^2(x_i,y_j) = \sum_{j=1}^{k} \frac{\lambda_j w_{ij}^2}{\sigma_{ii}}.
	\]
	since $x_i = w_{i1}y_1 + \cdots + w_{im}y_m$, thus we know that $\rho_{i\cdot 1,\cdots,m}^2 = 1$, i.e.
	\[
		\sum_{j=1}^{m} \rho^2(x_i,y_j) = \sum_{j=1}^{m} \frac{\lambda_j w_{ij}^2}{\sigma_{ii}} = 1
	\]
	then we get 
	\[
		\sum_{j=1}^{m} \lambda_j w_{ij}^2 = \sigma_{ii},~i.e.~\mathbf{W}^T\Lambda\mathbf{W} = \mathbf{\Sigma}.
	\]
	
\item the influence of the original variable $\mathbf{x}$ to the principal component variable $\mathbf{y}$

	Consider 
	\[
		\mathbf{y} = \mathbf{W}^Tx 
	\]
	i.e.
	\[
		y_i = w_{1i} x_1 + \cdots + w_{mi} x_m.
	\]
	we call $w_{ik}$ as the \textbf{load} of the k-th principal component $y_k$ on original variable $x_i$, it measures the influence of $x_i$ on $y_k$.
\end{enumerate}
\subsection{Algorithm}

Do \{

	\qquad eigenvalue decomposition $\mathbf{W}^T\Lambda\mathbf{W} = \mathbf{\Sigma} = \mathbf{Cov(x)}$
	\vskip 3 mm
	\qquad reorder $\lambda_{(1)}<\cdots<\lambda_{(m)},~\mathbf{w}_{(1)}<\cdots<\mathbf{w}_{(m)}$

\}

\subsection{PCA of correlation coefficient matrix}
Sometimes not suitable for directly starting PCA from the covariance matrix $\mathbf{Cov(x)}$. For example, the units of each variable $x_i$ are not all the same, or units are the same, but the difference in variance of each variable is very large, so that the the variables with small variance was almost  ignored, but the variables with large variance accounted for the largest proportion.

In these cases, we usually standardize the variables $\mathbf{x}$ first, and then start with the covariance matrix of the standardized variables to find the principal components.

The usual standardization transformation are

\[
	\mathbf{x}^* = (x_1^*,\cdots,x_m^*),~x_i^* = \frac{x_i - \mu_i}{\sqrt{\sigma_{ii}}},~i = 1,\cdots,p.
\]
Obviously, the covariance matrix of $\mathbf{x}^*$ is exactly the correlation coefficient matrix $\mathbf{R}$ of $\mathbf{x}$, i.e.
\begin{tiny}
\[	
	\mathbf{R} =
	\begin{pmatrix}
		\rho(x_i,x_j)
	\end{pmatrix}
	_{m\times m}
	= 	
	\begin{pmatrix}
		\frac{\mathbf{Cov}(x_i,x_j)}{\sqrt{\mathbf{Var}(x_i)\mathbf{Var}(x_j)}}
	\end{pmatrix}
	_{m\times m}
	= 	
	\begin{pmatrix}
		\mathbf{Cov}\left(\frac{x_i-\mu_i}{\sqrt{\sigma_{ii}}},\frac{x_j-\mu_j}{\sqrt{\sigma_{jj}}}\right)
	\end{pmatrix}
	_{m\times m}
	=
	\mathbf{Cov}(\mathbf{x^*})
\]
\end{tiny}
\[
	\mathbf{\Sigma}^* = \mathbf{Cov}(\mathbf{x^*}) = \mathbf{R}
\]
we replace $\mathbf{\Sigma}=\mathbf{\Sigma}^*$, all analysis processes of PCA are the same with previous section.

 

% 

\chapter{Model selection and evaluation}



% \chapter{Regression}


\section{Generalized Linear Models}

The following are a set of methods intended for regression
\begin{equation}
	y = \theta_0 + \theta_1 x_1 + \theta_2 x_2 + \theta_p x_p  + \epsilon
\end{equation}
where
\begin{align*}
	\mathbf{E}(\epsilon)   &= 0\\
	\mathbf{Var}(\epsilon) &= \sigma^2>0.
\end{align*}


\noindent For statistical analysis, n observations are made
\[
(x_{i1},x_{i2},\cdots,x_{ip},y_i),i = 1,2,\cdots,n
\]
accordingly
\[
	y_i = \theta_0 + \theta_1 x_{i1} + \theta_2 x_{i2} + \theta_p x_{ip}  + \epsilon_i, i = 1,2,\cdots,n
\]
satisfy Gauss-Markov assumption 
\begin{align*}
	\mathbf{E}(\epsilon_i)  			&= 0\\
	\mathbf{Var}(\epsilon_i) 			&= \sigma^2, \sigma^2\in (0,\infty)\\
	\mathbf{Cov}(\epsilon_i,\epsilon_j) &= 0, i \neq j.
\end{align*}
if we denote
\begin{align*}
	\mathbf{y} =
				\begin{pmatrix}
				y_1\\
				y_2\\
				\vdots\\
				y_n
				\end{pmatrix}
,
	\mathbf{X} =
				\begin{pmatrix}
				1 		&x_{11} &x_{12} &\cdots &x_{1p}\\
				1 		&x_{21} &x_{22} &\cdots &x_{2p}\\
				\vdots	&\vdots		 &\vdots	  &\ddots  &\vdots     \\
				1 		&x_{n1} &x_{n2} &\cdots &x_{np}\\
				\end{pmatrix}
,
	\mathbf{\theta} =
				\begin{pmatrix}
				\theta_0\\
				\theta_1\\
				\vdots\\
				\theta_p\\
				\end{pmatrix}
,
	\mathbf{\epsilon} =
				\begin{pmatrix}
				\epsilon_1\\
				\epsilon_2\\
				\vdots\\
				\epsilon_n\\
				\end{pmatrix}.
\end{align*}
\noindent $X$ is called \textbf{design matrix}
\noindent We can get a compact form
\begin{align}
	\mathbf{y} &= \mathbf{X}\mathbf{\theta} + \mathbf{\epsilon}
\end{align}
with  Gauss-Markov assumption 
\begin{align}
	\mathbf{E}(\mathbf{\epsilon}) &= \mathbf{0}_n\\
	\mathbf{Cov}(\epsilon) &= \sigma ^2\mathbf{I}_n.
\end{align}
 



%-------------------------------------------------------------------------------------------------------------
\newpage
\subsection{Ordinary Least Squares}
\noindent model
\begin{equation}
\underset{\theta} {\min\,} {\| \mathbf{X}\mathbf{\theta} - \mathbf{y}\|_2^2} 
\end{equation}

\noindent we define the \textbf{cost function}
\begin{equation*}
	J(\theta) = \frac{1}{2} \| \mathbf{X}\mathbf{\theta} - \mathbf{y}\|_2^2
\end{equation*}

\noindent \textbf{Method I:}

\noindent We want to choose $\theta$ to minimize $J(\theta)$. Lets Consider the \textbf{GD} algorithm, which starts with some initial $\theta$, and repeatedly performs the update:
\begin{equation*}
	\theta_j = \theta_j - \alpha \frac{\partial}{\partial\theta_j}J(\theta), j = 0,1,\cdots,p
\end{equation*}
\noindent where $\alpha \text{ is the \textbf{learning rate}}$, and
\begin{align*}
	&\frac{\partial}{\partial\theta_0}J(\theta) = 2\sum_{i=1}^{n}[\theta_0 + \theta_1 x_{i1} + \cdots + \theta_p x_{ip}], \quad j = 0\\  
	&\frac{\partial}{\partial\theta_j}J(\theta) = 2\sum_{i=1}^{n}[\theta_0 + \theta_1 x_{i1} + \cdots + \theta_p x_{ip}]x_{ij}, j >0
\end{align*}

\noindent \textbf{Algorithm:}

Loop until convergence \{

	\qquad $\theta_j = \theta_j - \alpha \frac{\partial}{\partial\theta_j}J(\theta), j = 0,1,\cdots,p$

\}




\noindent \textbf{Method II:}

\noindent To minimizes $J(\theta)$, we set its derivatives to zero
\begin{align*}
			\nabla_{\theta} J(\theta) =  \mathbf{0}
\end{align*}
then we obtain the \textbf{norm function}
\begin{align*}
	\mathbf{X}^T(\mathbf{X}\mathbf{\theta} - \mathbf{y}) = \mathbf{0} \rightarrow \mathbf{X}^T\mathbf{X}\mathbf{\theta} = \mathbf{X}^T\mathbf{y}
\end{align*}
\noindent The necessary and sufficient condition for above normal equation have unique solution is $\mathbf{X}^T\mathbf{X}$ have inverse. Thus, the value of $\theta$ that minimizes $J(\theta)$ have closed form:
\[
\mathbf{\theta} = (\mathbf{X}^T\mathbf{X})^{-1}\mathbf{X}^T\mathbf{y}
\]


\noindent \textbf{Algorithm:}

Do \{

	\qquad $\mathbf{\theta} = (\mathbf{X}^T\mathbf{X})^{-1}\mathbf{X}^T\mathbf{y}$

\}

		


%-------------------------------------------------------------------------------------------------------------
\newpage
\subsection{Ridge Regression}
\noindent model 
\footnote{Ref: Notes on Regularized Least Squares Ryan M. Rifkin and Ross A. Lippert}

\begin{equation}
\underset{\mathbf{\theta}} {\min\,} {\| \mathbf{X}\mathbf{\theta} - \mathbf{y}\|_2^2} + \alpha \|\mathbf{\theta}\|_2^2
\end{equation}


\noindent we define the \textbf{cost function}
\begin{equation*}
	J(\mathbf{\theta}) = \frac{1}{2} \| \mathbf{X}\mathbf{\theta} - \mathbf{y} \|_2^2 + \frac{1}{2} \alpha\|\mathbf{\theta}\|_2^2
\end{equation*}


\noindent To minimizes $J(\theta)$, we set its derivatives to zero
\begin{align*}
			\nabla_{\theta} J(\theta) =  \mathbf{0}
\end{align*}
then we obtain the \textbf{norm function}
\begin{align*}
	\mathbf{X}^T(\mathbf{X}\mathbf{\theta} - \mathbf{y}) - \alpha \mathbf{\theta} = \mathbf{0} \rightarrow (\mathbf{X}^T\mathbf{X} + \alpha \mathbf{I})\mathbf{\theta} = \mathbf{X}^T\mathbf{y}
\end{align*}
\noindent The necessary and sufficient condition for above normal equation have unique solution is $\mathbf{X}^T\mathbf{X} + \alpha \mathbf{I}$ have inverse. Thus, the value of $\theta$ that minimizes $J(\theta)$ have closed form:
\[
\mathbf{\theta} = (\mathbf{X}^T\mathbf{X} + \alpha \mathbf{I})^{-1}\mathbf{X}^T\mathbf{y}
\]


\noindent \textbf{Algorithm:}

Do \{

	\qquad $\mathbf{\theta} = (\mathbf{X}^T\mathbf{X} + \alpha \mathbf{I})^{-1}\mathbf{X}^T\mathbf{y}$

\}



%-------------------------------------------------------------------------------------------------------------
\newpage
\subsection{Lasso}
\noindent model 
\footnote{
Ref: \href{https://en.wikipedia.org/wiki/Lasso_(statistics)}{https://en.wikipedia.org/wiki/Lasso\_(statistics)}\\
Lasso: regression shrinkage and selection via the lasso, by ROBERT TIBSHIRANI
}

\begin{align}
	\underset{\mathbf{\theta}}{\min\,}  \frac{1}{n} \|\mathbf{X}\mathbf{\theta} - y\|_2 ^2 + \alpha\|\mathbf{\theta}\|_1
\end{align}

\noindent It is standard to work with centered variables. Additionally, the covariates are typically standardized $\sum _{i=1}^{n}x_{ij}^{2} = 1$ so that the solution does not depend on the measurement scale.

\noindent we define the \textbf{cost function}
\begin{equation*}
	J(\mathbf{\theta}) = \frac{1}{n}\| \mathbf{X}\mathbf{\theta} - \mathbf{y} \|_2^2 + \alpha\|\mathbf{\theta}\|_1
\end{equation*}

\noindent Some basic properties of the lasso estimator can now be considered.

\noindent Assuming first that the covariates are orthonormal so that $X^{T}X=I$. To minimizes $J(\theta)$, using subgradient methods it can be shown that
\begin{align*}
	\nabla_{\theta} J(\theta) =  \mathbf{0}
\end{align*}
then we get the shrinkage
\begin{equation*}
	\mathbf{\theta}_j = S_1^{\lambda}(\mathbf{\theta}_j^{OLS}) 
	= sign(\mathbf{\theta}_j^{OLS}) max \left(|\mathbf{\theta}_j^{OLS}| - n \alpha, 0 \right),
	\text{where } \mathbf{\theta}^{OLS} = (\mathbf{X}^T\mathbf{X})^{-1}\mathbf{X}^T\mathbf{y}
\end{equation*}


\noindent \textbf{Algorithm:}

Loop until convergence\{

	\qquad $\mathbf{\theta}^{OLS} = (\mathbf{X}^T\mathbf{X})^{-1}\mathbf{X}^T\mathbf{y}$

	\qquad $\mathbf{\theta_j} = S_1^{\lambda}(\mathbf{\theta}_j^{OLS}), j = 0,1,\cdots,p$

\}








% \subsection{Multi-task Lasso}
%  model
%  \begin{align}
% \underset{w}{min\,} & {\frac{1} {2n_{samples}} \|X W - Y\|_{Fro} ^ 2 + \alpha \|W\|_{21}}\nonumber\\
%                     &\|A\|_{Fro} = \sqrt{\sum_{ij} a_{ij}^2}\nonumber\\
%                     &\|A\|_{2 1} = \sum_i \sqrt{\sum_j a_{ij}^2}\nonumber
% \end{align}



% \subsection{Elastic Net}
%  model
% \begin{align}
% \underset{w}{min\,} { \frac{1}{2n_{samples}} \|X w - y\|_2 ^ 2 + \alpha \rho \|w\|_1 +
% \frac{\alpha(1-\rho)}{2} \|w\|_2 ^ 2}\nonumber 
% \end{align}


% \subsection{Multi-task Elastic Net}
%  model
% \begin{align}
% \underset{W}{min\,} { \frac{1}{2n_{samples}} \|X W - Y\|_{Fro}^2 + \alpha \rho \|W\|_{2 1} +
% \frac{\alpha(1-\rho)}{2} \|W\|_{Fro}^2}\nonumber
% \end{align}
 


% \subsection{Least Angle Regression}
% model

% Least-angle regression (LARS) is a regression algorithm for high-dimensional data, developed by Bradley Efron, Trevor Hastie, Iain Johnstone and Robert Tibshirani. LARS is similar to forward stepwise regression. At each step, it finds the predictor most correlated with the response. When there are multiple predictors having equal correlation, instead of continuing along the same predictor, it proceeds in a direction equiangular between the predictors.

% \subsection{LARS Lasso}
% model

% The algorithm is similar to forward stepwise regression, but instead of including variables at each step, the estimated parameters are increased in a direction equiangular to each one’s correlations with the residual.

% \subsection{Orthogonal Matching Pursuit (OMP)}
%  model
% \begin{align}
%     &\text{arg\,min\,} \|y - X\gamma\|_2^2 \text{ s.t } \|\gamma\|_0 \leq n_{nonzero\_coefs}~or~\nonumber\\
%     &\text{arg\,min\,} \|\gamma\|_0 \qquad ~\text{ s.t } \|y-X\gamma\|_2^2 \
% \leq \text{tol}\nonumber
% \end{align}

% \subsection{Bayesian Regression}
%  model
% \begin{align}
%     &p(y|X,w,\alpha) = \mathcal{N}(y|X w,\alpha)\nonumber\\
% \end{align}

 
% \subsection{Logistic regression}
%  model
% \begin{align}
%     &l_1: ~\underset{w, c}{min\,} \frac{1}{2}w^T w + C \sum_{i=1}^n \log(\exp(- y_i (X_i^T w + c)) + 1)~or \nonumber\\
%     &l_2:~\underset{w, c}{min\,} \|w\|_1 + C \sum_{i=1}^n \log(\exp(- y_i (X_i^T w + c)) + 1) \nonumber
% \end{align}
 

%  \subsection{Stochastic Gradient Descent - SGD}
% model

% Stochastic gradient descent is a simple yet very efficient approach to fit linear models. It is particularly useful when the number of samples (and the number of features) is very large


% \subsection{Perceptron}
% model

% The Perceptron is another simple algorithm suitable for large scale learning. By default:
% \begin{itemize}
% \item It does not require a learning rate.
% \item It is not regularized (penalized).
% \item It updates its model only on mistakes.
% \end{itemize}


% \subsection{Passive Aggressive Algorithms}
% model

% The passive-aggressive algorithms are a family of algorithms for large-scale learning. They are similar to the Perceptron in that they do not require a learning rate. However, contrary to the Perceptron, they include a regularization parameter C.

% \subsection{Robustness regression: outliers and modeling errors}
%  model

% Robust regression is interested in fitting a regression model in the presence of corrupt data: either outliers, or error in the model.

 
% \subsection{Polynomial regression: extending linear models with basis functions}
%  model
% \begin{align}
%     &\hat{y}(w, x) = w_0 + w_1 x_1 + w_2 x_2 + w_3 x_1 x_2 + w_4 x_1^2 + w_5 x_2^2 \nonumber\\
%     &z = [x_1, x_2, x_1 x_2, x_1^2, x_2^2]~\text{creating a new variable list, then get the linear model}\nonumber\\
%     &\hat{y}(w, x) = w_0 + w_1 z_1 + w_2 z_2 + w_3 z_3 + w_4 z_4 + w_5 z_5 \nonumber 
% \end{align}

 
% \section{Linear and Quadratic Discriminant Analysis}
% This section is going to talk about LDA and QDA

% Linear Discriminant Analysis and Quadratic Discriminant Analysis are two classic classifiers, with, as their names suggest, a linear and a quadratic decision surface, respectively.

% These classifiers are attractive because they have closed-form solutions that can be easily computed, are inherently multiclass, have proven to work well in practice and have no hyperparameters to tune.


% \subsection{Dimensionality reduction using Linear Discriminant Analysis}
%  model

% LDA can be used to perform supervised dimensionality reduction, by projecting the input data to a linear subspace consisting of the directions which maximize the separation between classes. The dimension of the output is necessarily less than the number of classes, so this is a in general a rather strong dimensionality reduction, and only makes senses in a multiclass setting.



% \subsection{Mathematical formulation of the LDA and QDA classifiers}
%  model

% Both LDA and QDA can be derived from simple probabilistic models which model the class conditional distribution of the data $P(X|y=k)$ for each class $k$. Predictions can then be obtained by using Bayes' rule, and we select the class k which maximizes this conditional probability.
% \begin{align}
%     &P(y=k | X) = \frac{P(X | y=k) P(y=k)}{P(X)} = \frac{P(X | y=k) P(y = k)}{ \sum_{l} P(X | y=l) \cdot P(y=l)}\nonumber\\
% \end{align}


% \subsection{Mathematical formulation of LDA dimensionality reduction}
%  model

% \begin{align}
%     \text{rescale the data: }&X^* = D^{-1/2}U^t X\text{ with }\Sigma = UDU^t, \Sigma = cov(X)\nonumber\\
% \end{align}

% \subsection{Shrinkage}
%  model

% Shrinkage is a tool to improve estimation of covariance matrices in situations where the number of training samples is small compared to the number of features. 

% \subsection{Estimation algorithms}
%  model

% The default solver is SVD. It can perform both classification and transform, and it does not rely on the calculation of the covariance matrix. This can be an advantage in situations where the number of features is large. However, the SVD solver cannot be used with shrinkage.


% \section{Support Vector Machines}
% Support vector machines (SVMs) are a set of supervised learning methods used for classification, regression and outliers detection.
 
% \subsection{Classification}
%  model

%     \subsubsection{Multi-class classification}
%      model

%     \subsubsection{Unbalanced problems}
%      model

% \subsection{Regression}
%  model

% The method of Support Vector Classification can be extended to solve regression problems. This method is called Support Vector Regression.


% \subsection{Density estimation, novelty detection}
%  model
 
% One-class SVM is used for novelty detection, that is, given a set of samples, it will detect the soft boundary of that set so as to classify new points as belonging to that set or not. 

% \subsection{Complexity}
%  model
 
% Support Vector Machines are powerful tools, but their compute and storage requirements increase rapidly with the number of training vectors.

% $O(n_{features} \times n_{samples}^2)$ —— $O(n_{features} \times n_{samples}^3)$ 

% \subsection{Kernel functions}
%  model
 
% The kernel function can be any of the following:
% \begin{itemize}
% \item linear: $\langle x, x^T\rangle$.
% \item polynomial: $(\gamma \langle x, x^T\rangle + r)^d$. 
% \item rbf: $\exp(-\gamma \|x-x^T\|^2)$.
% \item sigmoid $(\tanh(\gamma \langle x,x^T\rangle + r))$.
% \item custom Kernels
% \end{itemize}

% \subsection{Mathematical formulation}
% A support vector machine constructs a hyper-plane or set of hyper-planes in a high or infinite dimensional space, which can be used for classification, regression or other tasks. 


% \subsubsection{SVC}
% Given training vectors $x_i \in \mathbb{R}^p, i=1,\dots,n$, in two classes, and a vector $y \in \{1, -1\}^n$, SVC solves the following primal problem:

% \begin{align}
%     \min_{w, b,\zeta} & \frac{1}{2} w^T w + C \sum_{i=1}^{n} \zeta_i\nonumber\\
%                       &\text{ s.t }  y_i (w^T \phi (x_i) + b) \geq 1 - \zeta_i\nonumber\\
%                       &\zeta_i \geq 0, i=1,\dots,n \nonumber
% \end{align}

% dual is

% \begin{align}
%     \min_{\alpha} &\frac{1}{2} \alpha^T Q \alpha - e^T \alpha\nonumber\\
%     &\text{ s.t } y^T \alpha = 0\nonumber\\
%     & 0 \leq \alpha_i \leq C, i=1,\dots,n \nonumber\\
% \end{align}


% \subsubsection{NuSVC}

% We introduce a new parameter $\nu$ which controls the number of support vectors and training errors. The parameter $\nu \in (0,
% 1]$ is an upper bound on the fraction of training errors and a lower bound of the fraction of support vectors.


% \subsubsection{SVR}

 

% Given training vectors $x_i \in \mathbb{R}^p, i=1,\dots,n$, and a vector $y \in \mathbb{R}^n \varepsilon$-SVR solves the following primal problem:

% \begin{align}
%     \min_ {w, b, \zeta, \zeta^*}& \frac{1}{2} w^T w + C \sum_{i=1}^{n} (\zeta_i + \zeta_i^*)\nonumber\\
%                       &\textrm { s.t }  y_i - w^T \phi (x_i) - b \leq \varepsilon + \zeta_i\nonumber\\
%                       & w^T \phi (x_i) + b - y_i \leq \varepsilon + \zeta_i^*\nonumber\\
%                       & \zeta_i, \zeta_i^* \geq 0, i=1,\dots, n\nonumber
% \end{align}

% dual is 

% \begin{align}
%     \min_{\alpha, \alpha^*} &\frac{1}{2} (\alpha - \alpha^*)^T Q (\alpha - \alpha^*) + \varepsilon e^T (\alpha + \alpha^*) - y^T (\alpha - \alpha^*)\nonumber\\
%                       &\textrm {s.t } e^T (\alpha - \alpha^*) = 0\nonumber\\
%                       &0 \leq \alpha_i, \alpha_i^* \leq C, i=1,\dots, n\nonumber
% \end{align}



\chapter{Clustering}


\newpage
\section{\textit{k}-means}


\subsection{Introduction}

\textit{k}-means clustering is a simple and elegant approach for partitioning a
data set $s$ into $k$ distinct, non-overlapping clusters. To perform \textit{k}-means clustering, we must first specify the desired number of clusters $k$, then the \textit{k}-means algorithm will assign each observation to exactly one of the $k$ clusters.


\subsection{Description}
In mathematics, that is we want to decompose $S = \{\boldsymbol x ^{(j)}\}_{i=1}^{n},\boldsymbol x ^{(j)}\in\mathbb{R}^m$ into $k$ class $C_1,C_2,\cdot,C_k$, s.t.
\[
	S = \bigsqcup _{i=1}^{k} C_i.
\]
The idea behind \textit{k}-means clustering is that a good clustering is one for which the within-cluster variation is as small as possible.

\noindent model 
\footnote{
	Ref: \href{https://en.wikipedia.org/wiki/k-means_clustering\#description}{https://en.wikipedia.org/wiki/k-means\_clustering\#description}\\
	\qquad $\mathbf{k}$-means: some methods for classification and analysis of multivariate observations,j. macqueen
}
\begin{align*}
	\underset{\mathbf{\mu}^{(1)},\cdots,\mathbf{\mu}^{(k)}}{\min\,} \sum_{i=1}^{k} \sum_{j=1}^{n} \chi_i^j \|\boldsymbol x^{(j)} - \mathbf{\mu}^{(i)} \|_2^2
\end{align*}
where 
\begin{align*}
	\chi_i^j = 
	\begin{cases}
        &1, \text{ if } \|\boldsymbol x^{(j)}-\mathbf{\mu^{(i)}}\|_2^2 = \min_s \|\boldsymbol x^{(j)}-\mathbf{\mu^{(s)}}\|\\
        &0, \text{ else. }
      \end{cases}
\end{align*}
\noindent We define the \textbf{cost function}
\begin{equation*}
	J(\mathbf{\mu}) = \sum_{i=1}^{k} \sum_{j=1}^{n} \chi_i^j \|\boldsymbol x^{(j)} - \mathbf{\mu}^{(i)} \|_2^2,\quad \mathbf{\mu} = (\mathbf{\mu}^{(1)},\cdots,\mathbf{\mu}^{(k)})
\end{equation*}
\noindent To minimizes $J(\mathbf{\mu}^{(1)},\cdots,\mathbf{\mu}^{(k)})$, we set its derivatives to zero 
\begin{align*}
	\nabla_{\mathbf{\mu}} J(\mu) =  \mathbf{0} .
\end{align*}
then we get
\begin{align*}
	\mathbf{\mu}^{(i)} = \frac{\sum_{j = 1}^{n}\chi_i^j \boldsymbol x^{(j)} }{\sum_{j=1}^{n}\chi_i^j}.
\end{align*}


\subsection{Algorithm}

\noindent \textbf{algorithm:}

Do \{

\quad 1. Initialize cluster centroids $\mathbf{\mu}_1,\cdots,\mathbf{\mu}_k$ randomly

\quad 2. Loop until convergence \{

	\qquad for each i \{
		$\chi_i^j = 
			\begin{cases}
			1, \text{ if } \|\boldsymbol x^{(j)}-\mathbf{\mu^{(i)}}\|_2^2 = \min_s \|\boldsymbol x^{(j)}-\mathbf{\mu^{(s)}}\|\\
			0, \text{ else.}
			\end{cases}
		$
		\}

	\qquad for each j \{$\mathbf{\mu}^{(i)} = \frac{\sum_{j = 1}^{n}\chi_i^j \boldsymbol x^{(j)} }{\sum_{j=1}^{n}\chi_i^j}.$\}

\}




\newpage 
\section{EM}

\subsection{Introduction}

	In statistics, an \textbf{Expectation-Maximization} (EM) algorithm is an iterative method to find maximum likelihood or maximum a posteriori (MAP) estimates of parameters in statistical models, where the model depends on unobserved latent variables. 

	The EM iteration alternates between performing an expectation (E) step, which creates a function for the expectation of the log-likelihood evaluated using the current estimate for the parameters, and a maximization (M) step, which computes parameters maximizing the expected log-likelihood found on the E step. These parameter-estimates are then used to determine the distribution of the latent variables in the next E step.


	% The EM algorithm is used to find (locally) maximum likelihood parameters of a statistical model in cases where the equations cannot be solved directly. Typically these models involve latent variables in addition to unknown parameters and known data observations. That is, either missing values exist among the data, or the model can be formulated more simply by assuming the existence of further unobserved data points.

	%  The EM algorithm proceeds from the observation that the following is a way to solve these two sets of equations numerically. One can simply pick arbitrary values for one of the two sets of unknowns, use them to estimate the second set, then use these new values to find a better estimate of the first set, and then keep alternating between the two until the resulting values both converge to fixed points. It's not obvious that this will work at all, but it can be proven that in this context it does, and that the derivative of the likelihood is (arbitrarily close to) zero at that point, which in turn means that the point is either a maximum or a saddle point. In general, multiple maxima may occur, with no guarantee that the global maximum will be found. Some likelihoods also have singularities in them, i.e., nonsensical maxima.

\subsection{Description}


Given the statistical model which generates a set $\boldsymbol x$ of observed data, a set of unobserved latent data or missing values $\mathbf{Z}$, and a vector of unknown parameters $\boldsymbol\theta$, along with a likelihood function $L(\boldsymbol\theta|\boldsymbol x, \mathbf{Z}) = p(\boldsymbol x, \mathbf{Z} | \boldsymbol\theta)$, the maximum likelihood estimate (MLE) of the unknown parameters is determined by the marginal likelihood of the observed data.
\begin{align}
	\label{ep:em_L}
	L(\boldsymbol \theta | \boldsymbol x) = p(\boldsymbol x| \boldsymbol \theta) = \int p(\boldsymbol x,\mathbf{Z}| \boldsymbol \theta) d\mathbf{Z}
\end{align}
The objective of the algorithm is to find the parameter $\boldsymbol \theta$, which maximizes the likelihood function $L(\boldsymbol \theta | \boldsymbol x)$ of the observed $\boldsymbol x$. Unfortunately, its quantity is often intractable since the random variable $\mathbf{Z}$ is hidden, if $\mathbf{Z}$ is a sequence of events, so that the number of values grows exponentially with the sequence length, making the exact calculation of the sum extremely difficult. 

The EM algorithm seeks to find the MLE of the marginal likelihood by iteratively applying these two steps:
\begin{itemize}
	\item[E:]  Expectation step

		Calculate the expected value of the log likelihood function, with respect to the conditional distribution of $\mathbf {Z}$ given $\boldsymbol x$  under the current estimate of the parameters $\boldsymbol\theta^{(t)}$:
	\begin{equation*}
		Q(\boldsymbol\theta|\boldsymbol\theta^{(t)}) = \operatorname{E}_{\mathbf{Z}|\boldsymbol x,\boldsymbol\theta^{(t)}}\left[ \log L (\boldsymbol\theta| \boldsymbol x,\mathbf{Z})  \right] \,
	\end{equation*}
	\item[M:] Maximization step

		Find the parameters $\boldsymbol\theta$ that maximize this quantity:
	\begin{equation*}
		\boldsymbol\theta^{(t+1)} = arg\underset{\boldsymbol\theta}{\max} \ Q(\boldsymbol\theta| \boldsymbol\theta^{(t)})
	\end{equation*}
\end{itemize}
The typical models to which EM is applied uses $\mathbf{Z}$ as a latent variable indicating membership in one of a set of groups.
\begin{enumerate}
	\item The observed data points $\boldsymbol x$ may be discrete (finite or countably) or continuous. Associated with each data point may be a vector of observations.

	\item The missing values (or latent variables) $\mathbf{Z}$ are discrete, drawn from a fixed number of values, and with one latent variable per observed unit.

	\item The parameters $\boldsymbol\theta$ are continuous, and are of two kinds: Parameters that are associated with all data points, and those associated with a specific value of a latent variable.
\end{enumerate}


\subsection{Properties}
Speaking of an E-step is a bit of a misnomer. What are calculated in the first step are the fixed, data-dependent parameters of the function $Q$. Once the parameters of $Q$ are known, it is fully determined and is maximized in the second M-step of an EM algorithm.

Although an EM iteration does increase the observed data likelihood function, no guarantee exists that the sequence converges to a maximum likelihood estimator. For multimodal distributions, this means that an EM algorithm may converge to a local maximum of the observed data likelihood function, depending on starting values. A variety of heuristic or metaheuristic approaches exist to escape a local maximum, such as random-restart hill climbing (starting with several different random initial estimates θ(t)), or applying simulated annealing methods.

EM is especially useful when the likelihood is an exponential family: the E-step becomes the sum of expectations of sufficient statistics, and the M-step involves maximizing a linear function. In such a case, it is usually possible to derive closed-form expression updates for each step, using the Sundberg formula (published by Rolf Sundberg using unpublished results of Per Martin-Lof and Anders Martin-Lof).

The EM method was modified to compute maximum a posteriori (MAP) estimates for Bayesian inference in the original paper by Dempster, Laird, and Rubin.

Other methods exist to find maximum likelihood estimates, such as gradient descent, conjugate gradient, or variants of the Gauss–Newton algorithm. Unlike EM, such methods typically require the evaluation of first and/or second derivatives of the likelihood function.


\subsection{Proof of correctness}

Expectation-maximization works to improve $Q(\boldsymbol\theta|\boldsymbol\theta^{(t)})$ rather than directly improving $\log p(\boldsymbol x|\boldsymbol\theta)$. Here is shown that improvements to the former imply improvements to the latter.

For any  $\mathbf{Z}$  with non-zero probability   $p(\mathbf{Z}|\boldsymbol x,\boldsymbol\theta)$, we can write
\[
	\log p(\boldsymbol x|\boldsymbol\theta) = \log p(\boldsymbol x,\mathbf{Z}|\boldsymbol\theta) - \log p(\mathbf{Z}|\boldsymbol x,\boldsymbol\theta),~\text{since }p(\boldsymbol x,\mathbf{Z}|\boldsymbol\theta) =p(\boldsymbol x|\boldsymbol\theta) p(\mathbf{Z}|\boldsymbol x,\boldsymbol\theta)
\]
We take the expectation over possible values of the unknown data  $\mathbf{Z}$  under the current parameter estimate $\boldsymbol\theta^{(t)}$ by multiplying both sides by $p(\mathbf{Z}| \boldsymbol x,\boldsymbol\theta^{(t)})$ and summing (or integrating) over $\mathbf{Z}$. The left-hand side is the expectation of a constant, so we get
\begin{scriptsize}
\begin{align*}
	\log p(\boldsymbol x|\boldsymbol\theta) &
	= \sum_{\mathbf{Z}} p(\mathbf{Z}|\boldsymbol x,\boldsymbol\theta^{(t)}) \log p(\boldsymbol x,\mathbf{Z}|\boldsymbol\theta)
	- \sum_{\mathbf{Z}} p(\mathbf{Z}|\boldsymbol x,\boldsymbol\theta^{(t)}) \log p(\mathbf{Z}|\boldsymbol x,\boldsymbol\theta) \\
	& = Q(\boldsymbol\theta|\boldsymbol\theta^{(t)}) + H(\boldsymbol\theta|\boldsymbol\theta^{(t)})
\end{align*}
\end{scriptsize}
This last equation holds for any value of $\boldsymbol {\theta}$ including 
$\boldsymbol \theta = \boldsymbol\theta^{(t)}$,
\[
	\log p(\boldsymbol x|\boldsymbol\theta^{(t)})
	= Q(\boldsymbol\theta^{(t)}|\boldsymbol\theta^{(t)}) + H(\boldsymbol\theta^{(t)}|\boldsymbol\theta^{(t)})
\]
and subtracting this last equation from the previous equation gives
\begin{scriptsize}
\[
	\log p(\boldsymbol x|\boldsymbol\theta) - \log p(\boldsymbol x|\boldsymbol\theta^{(t)})
	= Q(\boldsymbol\theta|\boldsymbol\theta^{(t)}) - Q(\boldsymbol\theta^{(t)}|\boldsymbol\theta^{(t)}) +H(\boldsymbol\theta|\boldsymbol\theta^{(t)})-H(\boldsymbol\theta^{(t)}|\boldsymbol\theta^{(t)})
\]
\end{scriptsize}
However, Gibbs' inequality tells us that $H(\boldsymbol\theta|\boldsymbol\theta^{(t)}) \ge H(\boldsymbol\theta^{(t)}|\boldsymbol\theta^{(t)})$, so we can conclude that
\[
	\log p(\boldsymbol x|\boldsymbol\theta) - \log p(\boldsymbol x|\boldsymbol\theta^{(t)})
	\ge Q(\boldsymbol\theta|\boldsymbol\theta^{(t)}) - Q(\boldsymbol\theta^{(t)}|\boldsymbol\theta^{(t)})
\]
In words, choosing  $\boldsymbol {\theta}$ to improve $Q(\boldsymbol\theta|\boldsymbol\theta^{(t)})$ beyond $Q(\boldsymbol\theta^{(t)}|\boldsymbol\theta^{(t)})$ cannot cause $\log p(\boldsymbol x|\boldsymbol\theta)$ to decrease below $\log p(\boldsymbol x|\boldsymbol\theta^{(t)})$, and so the marginal likelihood of the data is non-decreasing.

The EM algorithm can be viewed as two alternating maximization steps, that is, as an example of coordinate ascent. Consider the function:
\begin{scriptsize}
\begin{align*}
	\displaystyle F(q,\theta ):
	&=E_{q}[\log L(\theta | x,Z)]+H(q)\\
	&=E_{q}[\log L(\theta | x,Z)]+E_q[-\log q(Z)]\\
	&=E_{q}[\log L(\theta | x,Z)-\log q(Z)]\\
	&=E_{q}[\log p(x,Z | \theta)-\log q(Z)]\\
	&=E_{q}[\log (p(Z|x,\theta) p(x |\theta))-\log q(Z)]\\
	&=E_{q}[\log p(Z|x,\theta) + \log p(x |\theta)-\log q(Z)]\\
	&=E_{q}\left[\log \frac{p(Z|x,\theta)}{q(Z)}+\log p(x |\theta)\right]\\
	&=E_{q}\left[\log \frac{p(Z|x,\theta)}{q(Z)}\right]+\log p(x |\theta)\\
	&=E_{q}\left[\log \frac{p(Z|x,\theta)}{q(Z)}\right]+\log L(\theta|x)\\
	&=\sum _{z}q(z)\,\log{\frac {p(z|x,\theta)}{q(z)}}+\log L(\theta|x)
\end{align*}
\end{scriptsize}
where $q$ is an arbitrary probability distribution over the unobserved data $Z$ and $H(q)$ is the \textbf{entropy}
	\footnote{Shannon defined the entropy $Η$ of a discrete random variable $\boldsymbol x$ with possible values $\{x_1,\cdots,x_n\}$ and probability mass function $p(X)$ as: $H(X)=E[-\log p(X)]$}
of the distribution $q$. This function can be written as
\begin{scriptsize}
\begin{align*}
 	\displaystyle F(q,\theta )
 	=-D_{\mathrm {KL} }\big(q{~\big\|~}p_{Z|X}(\cdot | x,\theta )\big)+\log L(\theta|x)=\sum _{z}q(z)\,\log{\frac {p(z|x,\theta)}{q(z)}}+\log L(\theta|x)
\end{align*}
\end{scriptsize}
where $\displaystyle p_{Z|X}(\cdot |x;\theta )$ is the conditional distribution of the unobserved data given the observed data $x$ and $D_{KL}$ is the \textbf{Kullback--Leibler divergence}
	\footnote{For discrete probability distributions $P$ and $Q$, the Kullback–Leibler divergence from $Q$ to $P$ is defined to be $\displaystyle D_{\mathrm {KL} }(P\|Q)=-\sum _{i}P(i)\,\log {\frac {Q(i)}{P(i)}}$
	}.

\subsection{Algorithm}
Then the steps in the EM algorithm may be viewed as:
 
\noindent \textbf{algorithm:}

Loop until convergence \{

	\qquad E-step: \qquad $q^{(t+1)} = arg\max _{q}\ F(q,\theta ^{(t)})$
	\vskip 3 mm
	\qquad M-step: \qquad $\theta^{(t+1)}=arg\max_{\theta }\ F(q^{(t)},\theta)$

\}

% \subsection{Gaussian Mixture}
% 可以拓展高斯混合模型












\newpage
\section{\textit{k}-NN}
\subsection{Introduction}
$k$-nearest neighbors algorithm ($k$-NN) is a non-parametric method used for classification and regression. Its a type of instance-based learning, or lazy learning, where the function is only approximated locally and all computation is deferred until classification. The $k$-NN algorithm is among the simplest of all machine learning algorithms.

Both for classification and regression, a useful technique can be to assign weight to the contributions of the neighbors, so that the nearer neighbors contribute more to the average than the more distant ones. For example, a common weighting scheme consists in giving each neighbor a weight of $1/d$, where $d$ is the distance to the neighbor.

The neighbors are taken from a set of objects for which the class (for $k$-NN classification) or the object property value (for $k$-NN regression) is known. This can be thought of as the training set for the algorithm, though no explicit training step is required.


\subsection{Description}
$k$-NN is a non-parametric method used for classification and regression. In both cases, the input consists of the $k$ closest training examples in the feature space. The output depends on whether $k$-NN is used for classification or regression:
\begin{itemize}
\item In $k$-NN classification, the output is a class membership. An object is classified by a majority vote of its neighbors, with the object being assigned to the class most common among its $k$ nearest neighbors ($k$ is a positive integer, typically small).
\item In $k$-NN regression, the output is the property value for the object. This value is the average of the values of its $k$ nearest neighbors.
\end{itemize}
The training examples are vectors in a multidimensional feature space, each with a class label. The training phase of the algorithm consists only of storing the feature vectors and class labels of the training samples.

In the classification phase, k is a user-defined constant, and an unlabeled vector (a query or test point) is classified by assigning the label which is most frequent among the k training samples nearest to that query point.

A commonly used distance metric for continuous variables is Euclidean distance. For discrete variables, such as for text classification, another metric can be used, such as the overlap metric (or Hamming distance). In the context of gene expression microarray data, for example, k-NN has also been employed with correlation coefficients such as Pearson and Spearman. Often, the classification accuracy of k-NN can be improved significantly if the distance metric is learned with specialized algorithms such as Large Margin Nearest Neighbor or Neighbourhood components analysis.

In mathematics, $k$-NN can be stated as follow

Suppose that the training samples $(\boldsymbol x ^{(1)},\boldsymbol y ^{(1)}),\dots,(\boldsymbol x ^{(m)},\boldsymbol y ^{(m)})\in \mathbb{R}^{d}\times \mathbf{S}$ is known, $\mathbf{S} = \{s_1,\cdots,s_n\}$ is the class label. $(\boldsymbol x,\boldsymbol y)$ is the unlabeled data need to be classified, that is $X$ can be seen as a random variables in $\mathbf{R}^d$ and $Y$ is unknown. So that
 \[
 	X|Y=r\sim P_{r},~r\in\mathbf{S},P_{r} \text{ is probability distributions}
 \] 
Given some norm $\|\cdot \|$ on $\mathbb {R}^{d}$ and a point $x\in {\mathbb{R}}^{d}$. Reorder training data
 \[
 	(\tilde{\boldsymbol x} ^{(1)},\tilde{\boldsymbol y} ^{(1)}),\dots ,(\tilde{\boldsymbol x} ^{(m)},\tilde{\boldsymbol y} ^{(m)}),~s.t.~\|\tilde{\boldsymbol x} ^{(1)}-\boldsymbol x\|\leq \dots \leq \|\tilde{\boldsymbol x} ^{(m)}-\boldsymbol x\|
 \]
Define a positive small constant $k$. For all $\|\tilde{\boldsymbol x}^{(i)} - \boldsymbol x\| <k $, count the number of times each class appears, that is 
\[
	c_i=Card\{\tilde{\boldsymbol y}^{(j)} == s_i: j\in\{j:\|\tilde{\boldsymbol x}^{(j)} - \boldsymbol x\| <k\}\}, i=1,\cdots, n
\]
get pair $\{(s_1,c_1),\cdots,(s_n,c_n)\}$. The $k$-NN says that 
\[
	\boldsymbol y = s_i,~s.t.~n_i=\max_{i}\{c_i\}
\]

\noindent\textbf{Parameter selection}

The best choice of $k$ depends upon the data, Generally, larger values of $k$ reduce the effect of noise on the classification, but make boundaries between classes less distinct. A good $k$ can be selected by various heuristic techniques. The special case where the class is predicted to be the class of the closest training sample, i.e. when $k = 1$, is called the nearest neighbor algorithm.

The accuracy of the $k$-NN algorithm can be severely degraded by the presence of noisy or irrelevant features, or if the feature scales are not consistent with their importance. Much research effort has been put into selecting or scaling features to improve classification. A particularly popular approach is the use of evolutionary algorithms to optimize feature scaling. Another popular approach is to scale features by the mutual information of the training data with the training classes.

In binary (two class) classification problems, it is helpful to choose $k$ to be an odd number as this avoids tied votes. One popular way of choosing the empirically optimal ¥ in this setting is via bootstrap method.


\subsection{Properties}
k-NN is a special case of a variable-bandwidth, kernel density "balloon" estimator with a uniform kernel. A peculiarity of the $k$-NN algorithm is that it is sensitive to the local structure of the data.

The naive version of the algorithm is easy to implement by computing the distances from the test example to all stored examples, but it is computationally intensive for large training sets. Using an approximate nearest neighbor search algorithm makes k-NN computationally tractable even for large data sets. Many nearest neighbor search algorithms have been proposed over the years; these generally seek to reduce the number of distance evaluations actually performed.

k-NN has some strong consistency results. As the amount of data approaches infinity, the two-class k-NN algorithm is guaranteed to yield an error rate no worse than twice the Bayes error rate. Various improvements to the k-NN speed are possible by using proximity graphs.

For multi-class k-NN classification, Cover and Hart (1967) prove an upper bound error rate of
\[
	R^{*}\ \leq \ R_{k-\mathrm{NN} }\ \leq \ R^{*}\left(2-{\frac {nR^{*}}{n-1}}\right)
\]
where $R^{*}$ is the Bayes error rate (which is the minimal error rate possible), $R_{k-\mathrm{NN}}$ is the $k$-NN error rate, and $n$ is the number of classes in the problem. For $n=2$ and as the Bayesian error rate $R^{*}$ approaches zero, this limit reduces to "not more than twice the Bayesian error rate".

% \subsection{Proof of correctness}
\newpage
\subsection{Algorithm}
Do \{

	\noindent\qquad input training data $\{\boldsymbol x^{(i)},\boldsymbol y^{(i)}\}_{i=1}^{m}$, test data $\boldsymbol x$, 
	\vskip 3 mm
	\qquad$(\tilde{\boldsymbol x} ^{(1)},\tilde{\boldsymbol y} ^{(1)}),\dots ,(\tilde{\boldsymbol x} ^{(m)},\tilde{\boldsymbol y} ^{(m)}),~s.t.~\|\tilde{\boldsymbol x} ^{(1)}-\boldsymbol x\|\leq \dots \leq \|\tilde{\boldsymbol x} ^{(m)}-\boldsymbol x\|$
	\vskip 3 mm
	\qquad $c_i=Card\{\tilde{\boldsymbol y}^{(j)} == s_i: j\in\{j:\|\tilde{\boldsymbol x}^{(j)} - \boldsymbol x\| <k\}\}, i=1,\cdots, n$
	\vskip 3 mm
	\qquad$\boldsymbol y = s_i,~s.t.~n_i=\max_{i}\{c_i\}$

\}

 













\newpage
\section{Hierarchical Clustering}



\subsection{Introduction}
Hierarchical clustering (hierarchical cluster analysis or HCA) is a method of cluster analysis which seeks to build a hierarchy of clusters. 

Strategies for hierarchical clustering generally fall into two types
\begin{itemize}
\item \textbf{Agglomerative:} This is a "bottom up" approach: each observation starts in its own cluster, and pairs of clusters are merged as one moves up the hierarchy.

\item \textbf{Divisive:} This is a "top down" approach: all observations start in one cluster, and splits are performed recursively as one moves down the hierarchy.
\end{itemize}
In general, the merges and splits are determined in a greedy manner. The results of hierarchical clustering are usually presented in a dendrogram.

In the general case, the complexity of agglomerative clustering is $\mathcal{O}(n^{2}\log(n))$, which makes it too slow for large data sets. Divisive clustering with an exhaustive search is $\mathcal{O}(2^{n})$, which is even worse. However, for some special cases, optimal efficient agglomerative methods (of complexity $\mathcal{O}(n^{2}))$) are known: SLINK for single-linkage and CLINK for complete-linkage clustering.


\subsection{Description}

\textbf{Cluster dissimilarity}

In order to decide which clusters should be combined (for agglomerative), or where a cluster should be split (for divisive), a measure of dissimilarity between sets of observations is required. In most methods of hierarchical clustering, this is achieved by use of an appropriate metric (a measure of distance between pairs of observations), and a linkage criterion which specifies the dissimilarity of sets as a function of the pairwise distances of observations in the sets.


\noindent\textbf{Metric}

The choice of an appropriate metric will influence the shape of the clusters, as some elements may be close to one another according to one distance and farther away according to another.

Some commonly used metrics for hierarchical clustering are:

\begin{itemize}
\item \textbf{Minkowski distance:} $\mathbf{d}(\boldsymbol x^{(i)},\boldsymbol x^{(j)}) = \|\boldsymbol x^{(i)} - \boldsymbol x^{(j)}\|_\alpha$
	
	\begin{align*}
	&\text{Manhattan distance:} &\alpha = 1,\quad &\mathbf{d}(\boldsymbol x ^{(i)},\boldsymbol x ^{(j)}) = \sum_{k=1}^n|x_{k}^{(i)}-x_{k}^{(j)}|\\
	&\text{Euclidean distance:}  &\alpha = 2,\quad &\mathbf{d}(\boldsymbol x ^{(i)},\boldsymbol x ^{(j)}) = \sqrt{\sum_{k=1}^n(x_{k}^{(i)}-x_{k}^{(j)})^2}\\
	&\text{Chebyshev distance:}  &\alpha = \infty,\quad &\mathbf{d}(\boldsymbol x ^{(i)},\boldsymbol x ^{(j)}) = \max_{1\leq k \leq n}|x_{k}^{(i)}-x_{k}^{(j)}|
	\end{align*}

When all the variables have different units or the measurement ranges vary greatly, cannot be directly used Minkowski distance, Each variable should be normalized at the beginning.


\item \textbf{Mahalanobis distance:} $\mathbf{d}(\boldsymbol x ^{(i)},\boldsymbol x ^{(j)}) = \sqrt{(\boldsymbol x ^{(i)} - \boldsymbol x ^{(j)})^T\mathbf{\Sigma}^{-1} (\boldsymbol x ^{(i)} - \boldsymbol x ^{(j)})}$

\end{itemize}


\noindent\textbf{Similarity Coefficient}

Clustering analysis not only can be used to classify the samples, and also can be used to classify the variables. When it is used in classification, we offen use similarity coefficient to measure the similarity between variables.

we denote $c_{ij}$ as the similarity coefficient of variables $x_i$ and $x_j$, The definition of $c_{ij}$ generally satisfies the following conditions

\begin{itemize}
	\item[i] $c_{ij}=\pm 1\text{ iif } x_i = ax_j + b, a\neq 0, b\in\mathbb{R}$
	\item[ii] $|c_{ij}| \leq 1$
	\item[iii] $|c_{ij}| = c_{ji}$
\end{itemize}

The two most commonly used similarity coefficients are

\begin{enumerate}
	\item Angle Cosine
	\[
	 	c_{ij}=\cos \theta_{ij}=\frac{(\boldsymbol x ^{(i)})^T\boldsymbol x ^{(j)}}{[((\boldsymbol x ^{(i)})^T\boldsymbol x ^{(i)})(\boldsymbol x ^{(j)}^T\boldsymbol x ^{(j)})]^{1/2}}
	 \]
	
	where $\boldsymbol x\in\mathbb{R}^n$ is the observation of variable $x$, $\theta_{ij}$ is the included angle of observation $\boldsymbol x ^{(i)}$ and $\boldsymbol x ^{(j)}$.

	\item correlation coefficient
	\[
		c_{ij}=\rho_{ij}=\frac{(\boldsymbol x ^{(i)}-\boldsymbol\mu^{(i)})^T(\boldsymbol x ^{(j)}-
		\boldsymbol \mu^{(j)})}{[(\boldsymbol x ^{(i)}-\boldsymbol\mu^{(i)})^T(\boldsymbol x ^{(i)}-\boldsymbol\mu^{(i)})(\boldsymbol x ^{(j)}-\boldsymbol\mu^{(j)})^T(\boldsymbol x ^{(j)}-\boldsymbol\mu^{(j)})]^{1/2}}
	\]
Note that, if $\boldsymbol x ^{(i)},\boldsymbol x ^{(j)}$ is the normalized, then the angle cosine is the correlation coefficient.
\end{enumerate}

\noindent\textbf{Hierarchical clustering}

Hierarchical clustering is one of the most widely used clustering methods in clustering analysis. Its basic idea is 

Assume that $\boldsymbol x^{(k)} \in \mathbb{R}^n$ be samples, and denote that $d_{ij}=d(\boldsymbol x ^{(i)},\boldsymbol x ^{(j)})$ be the distance between samples $\boldsymbol x ^{(i)}$ and $\boldsymbol x ^{(j)}$, $C_1,C_2,\cdots$ be the classes. $D_{rs}$ be the distance between class $C_r$ and $C_s$.

\begin{enumerate}
	\item Make each sample $\boldsymbol x^{(k)}$ be a class.
	\item Define the distance $d_{ij}$ between samples $\boldsymbol x ^{(i)}, \boldsymbol x ^{(j)}$, and the distance $D_{rs}$ between classes $C_r, C_s$.
	\item The nearest two classes are merged into a new class, and calculate the distance between the new class and other classes.
	\item Combinate the two nearest classes repeatedly, each time a class is reduced until all the samples are merged into one class.
\end{enumerate}

With different class distance setting, we can get different hierarchical clustering algorithm. The common distance between classes as follow

\noindent \textbf{Distance Between Classes}
\begin{enumerate}
	\item \textbf{Minimum}
	\[
		D_{rs} = \min_{\boldsymbol x ^{(i)} \in C_r,\boldsymbol x ^{(j)} \in C_s} d_{ij}
	\]
	\item \textbf{Maximum}
	\[
		D_{rs} = \max_{\boldsymbol x ^{(i)} \in C_r,\boldsymbol x ^{(j)} \in C_s} d_{ij}
	\]
	\item \textbf{Mean}
	\[
		D_{rs} = \frac{1}{|C_r||C_s|}\sum_{\boldsymbol x ^{(j)} \in C_r}\sum_{\boldsymbol x ^{(j)} \in C_s}d_{ij}
	\]
	\item \textbf{Centroid}
	\[
		D_{rs} = \|c_{r}-c_{s}\|_2^2, \text{ where } c_{r} = \frac{1}{|C_r|}\sum_{\boldsymbol x ^{(i)}\in C_r}\boldsymbol x ^{(i)}, c_{s} = \frac{1}{|C_s|}\sum_{\boldsymbol x ^{(i)}\in C_s}\boldsymbol x ^{(i)}
	\]
\end{enumerate}
where $d$ is the chosen metric.

% \subsection{Properties}















% \subsection{Proof of correctness}

\subsection{Algorithm}

\noindent \textbf{algorithm:}

Do \{

	\qquad $C_k = \{\boldsymbol x^{(k)}\}, k = 1,\cdots,n$
	\vskip 1 mm
    \qquad $\mathbf{D}_{(0)} = [D_{rs}]$

	\vskip 1 mm
	\qquad For i From 1 to n-1 \{

		\vskip 1 mm
		\qquad \qquad $D_{rs} = \min_{i,j} \mathbf{D}_{i,j}>0$
		\vskip 1 mm
		\qquad \qquad $C_{new} = C_r \sqcup C_s$
		\vskip 1 mm
		\qquad \qquad $\mathbf{D}_{(i)} = \mathbf{D}_{(i-1)}$ delete r-th,s-th row, column
		\vskip 1 mm
		\qquad \qquad $\mathbf{D}_{(i)} = \mathbf{D}_{(i)}$ add a row, column at (n-i)-th with $D_{new,i},D_{i,new}$
		\vskip 1 mm
		\qquad \qquad relabel $C_k, k = 1,\cdots,n-i$ 	

	\qquad\}


\}











% \newpage
% \section{\textit{k}-NN}


% \subsection{Introduction}
% \subsection{Description}
% \subsection{Properties}
% \subsection{Proof of correctness}
% \subsection{Algorithm}


\chapter{Classification}




\section{Logistic Regression}

\subsection{Introduction}

Logistic regression was developed by statistician David Cox in 1958. In statistics, logistic regression is a regression model where the dependent variable is categorical. Logistic regression can be binomial, ordinal or multinomial.
\begin{itemize}
	\item Binomial or binary logistic regression deals with situations in which the observed outcome for a dependent variable can have only two possible types, "0" and "1". 
	\item Multinomial logistic regression deals with situations where the outcome can have three or more possible types that are not ordered. 
	\item Ordinal logistic regression deals with dependent variables that are ordered.
\end{itemize}
This work covers the case of a binary dependent variable, that is, where the output can take only two values, "0" and "1". The binary logistic model is used to estimate the probability of a binary response based on one or more predictor variables. It allows one to say that the presence of a risk factor increases the odds of a given outcome by a specific factor.


\subsection{Description}

\textbf{logistic function}
\[
	h (t)={\frac {1}{1+e^{-t}}}, t\in \mathbb{R}
\]
its derivative is
\[
	h'(t)= (1-h(t))h(t)
\]
its inverse
\[
	t = \log \frac{h(t)}{1-h(t)}, h(t) \in (0,1)
\]
Assume that $X\in\mathbb{R}^{m+1},Y\in\{0,1\},\boldsymbol \theta = (\theta_0,\theta_1,\cdot,\theta_m)^T$, $X,Y$ be random variable, $\boldsymbol x=(x_0,x_1,\cdots,x_m)^T, x_0=1, y \in \{0,1\}$ are samples of $X,Y$ respectively, and all samples are independent,  
\[
	Y|X,\boldsymbol \theta \sim P(Y| X,\boldsymbol\theta)
\]
Consider a generalized linear model function parameterized by $\boldsymbol\theta$
\[
	h(\boldsymbol\theta^T\boldsymbol x)=\frac {1}{1+e^{-\boldsymbol\theta^T \boldsymbol x}}
\]
If we assume that 
\[
	  P(y|\boldsymbol x,\boldsymbol\theta) = [h(\boldsymbol\theta^T\boldsymbol x)]^{y}[1-h(\boldsymbol\theta^T\boldsymbol x)]^{(1-y)}
\]
We define the loss functions as
\begin{align*}
	  L(\boldsymbol \theta) &= \prod_{i=1}^n  P(y^{(i)}|\boldsymbol x^{(i)} ,\boldsymbol\theta)\\
	  &= \prod _{i=1}^n [h(\boldsymbol\theta^T\boldsymbol x^{(i)} )]^{y^{(i)} }[1-h(\boldsymbol\theta^T\boldsymbol x^{(i)} )]^{(1-y^{(i)} )}
\end{align*}
Tt will be easier to maximize the log likelihood:
\[
	 J(\boldsymbol \theta) = \log L(\boldsymbol \theta) = \sum_{i=1}^n(y^{(i)} \log h(\boldsymbol\theta^T\boldsymbol x^{(i)} )  + (1-y^{(i)}) \log(1 - h(\boldsymbol\theta^T\boldsymbol x^{(i)}))  
\]
which can be maximized by gradient descent.
\begin{align*}
	 \frac{\partial J(\boldsymbol\theta)}{\partial\theta_j} 
	&=\sum_{i=1}^n [y^{(i)}\frac{1}{h(\boldsymbol \theta^T\boldsymbol x^{(i)})}\frac{\partial h(\boldsymbol \theta^T\boldsymbol x^{(i)})}{\partial \theta_j} + (1-y^{(i)})\frac{1}{1-h(\boldsymbol \theta^T\boldsymbol x^{(i)})}\frac{\partial (1-h(\boldsymbol \theta^T\boldsymbol x^{(i)}))}{\partial \theta_j} ]\\
	&=\sum_{i=1}^n [y^{(i)}\frac{1}{h(\boldsymbol \theta^T\boldsymbol x^{(i)})} - (1-y^{(i)})\frac{1}{1-h(\boldsymbol \theta^T\boldsymbol x^{(i)})}]\frac{\partial h(\boldsymbol \theta^T\boldsymbol x^{(i)})}{\partial \theta_j}\\
	&=\sum_{i=1}^n \frac{y^{(i)}-h(\boldsymbol \theta^T\boldsymbol x^{(i))}} {h(\boldsymbol \theta^T\boldsymbol x^{(i)})(1-h(\boldsymbol \theta^T\boldsymbol x^{(i)}))} h(\boldsymbol \theta^T\boldsymbol x^{(i)})(1-h(\boldsymbol \theta^T\boldsymbol x^{(i)})) \frac{\partial \boldsymbol \theta^T\boldsymbol x^{(i)}}{\partial \theta_j}\\
	&=\sum_{i=1}^n (y^{(i)}-h(\boldsymbol \theta^T\boldsymbol x^{(i)})) x_{j}^{(i)}
\end{align*}
that is
\[	
	\nabla _{\boldsymbol\theta} J(\boldsymbol\theta) = \sum_{i=1}^n (y^{(i)}-h(\boldsymbol \theta^T\boldsymbol x^{(i)})) (\boldsymbol x^{(i)})^T
\]
We update $\boldsymbol \theta$ by GD

\[	
	\boldsymbol\theta = \boldsymbol\theta - \alpha \nabla _{\boldsymbol\theta} J(\boldsymbol\theta)
\]





\subsection{Algorithm}
Loop until convergence \{

	\qquad $\nabla _{\boldsymbol\theta} J(\boldsymbol\theta) = \sum_{i=1}^n (y^{(i)}-h(\boldsymbol \theta^T\boldsymbol x^{(i)})) (\boldsymbol x^{(i)})^T$
	\vskip 1 mm
	\qquad $\boldsymbol\theta = \boldsymbol\theta - \alpha \nabla _{\boldsymbol\theta} J(\boldsymbol\theta)$

\}










\newpage 
\section{Softmax Regression}
In statistics, multinomial logistic regression is a classification method that generalizes logistic regression to multiclass problems. That is, it is a model that is used to predict the probabilities of the different possible outcomes of a categorically distributed dependent variable, given a set of independent variables, which may be real-valued, binary-valued, categorical-valued, etc.


\subsection{Introduction}
Multinomial logistic regression is used when the dependent variable in question is nominal and for which there are more than two categories. Nominal means category, which items that cannot be meaningfully ordered.

Multinomial logistic regression is a particular solution to the classification problem that assumes that a linear combination of the observed features and some problem-specific parameters can be used to determine the probability of each particular outcome of the dependent variable. The best values of the parameters for a given problem are usually determined from some training data.

\subsection{Description}

\textbf{Assumptions}



% \chapter{Deep Learning}

\subsection{Neural Network}

\subsection{Regularization in Deep Learning}

\subsection{Model Optimization}

\subsection{Deep Feed Forward Neural Network}

\subsection{Convolutional Neural Network}

\subsection{Recurrent Neural Networks}

\subsection{Methodology}



% \appendix
% \chapter{Appendix A}



\section{linear algebra}


\section{Probability and information theory}


\section{Numerical Calculation}

\end{document}









